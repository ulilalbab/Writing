\documentclass{article}
\usepackage[utf8]{inputenc}
\usepackage[english]{babel}
\usepackage{biblatex}
\usepackage{csquotes}
\addbibresource{references.bib}

\title{Pentingnya Kesadaran Akan Kesehatan Mental}
\author{Muhammad Ulil Albab }
\date{Agustus 2020}
\begin{document}
\maketitle

Bisa saja kita sering melihat orang yang dari luarnya nampak baik-baik saja, sehat dan tidak memiliki masalah dalam hidupnya. Namun pada kenyataannya semuanya bisa berbanding terbalik. Jika melihat kondisi di Indonesia sendiri, kesehatan mental sendiri masih dibilang hal yang baru dan jarang sekali diperbincangkan. Bahkan seringkali dianggap tabu atau cenderung dicari berbagai pembenaran dari lingkungan sekitar.  \par
Ada anggapan mungkin kurang ibadah, kurang bersyukur atau pembenaran lain yang terkadang malah memberikan tekanan sendiri bagi para orang yang sedang memiliki masalah mental. Masalah kesehatan mental harus dipahami sebagai masalah yang bisa menimpa siapa saja. Bisa jadi disebabkan oleh lingkungan kerja, kuliah atau bahkan keluarga. Pergi ke psikolog bisa menjadi langkah awal untuk membantu mengatasi berbagai masalah mental seperti \textit{panic attack}, \textit{anxiety disorder} atau gangguan mental lainnya.  

\section{Komunikasi Menjadi Kunci}
Menceritakan akan masalah yang sedang kita hadapi bisa menjadi langkah awal yang baik sepanjang kita bisa Mengkomunikasikan apa yang terjadi pada diri kita pada orang yang berkompenten \cite{ayuningtyas2018analisis}




\end{document}